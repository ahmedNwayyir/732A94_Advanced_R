%%%%%%%%%%%%%%%%%%%%%%%%%%%%%%%%%%%%%%%%%
% Beamer Presentation
% LaTeX Template
% Version 1.0 (10/11/12)
%
% This template has been downloaded from:
% http://www.LaTeXTemplates.com
%
% License:
% CC BY-NC-SA 3.0 (http://creativecommons.org/licenses/by-nc-sa/3.0/)
%
%%%%%%%%%%%%%%%%%%%%%%%%%%%%%%%%%%%%%%%%%

%----------------------------------------------------------------------------------------
%	PACKAGES AND THEMES
%----------------------------------------------------------------------------------------

\documentclass{beamer}

\mode<presentation> {

% The Beamer class comes with a number of default slide themes
% which change the colors and layouts of slides. Below this is a list
% of all the themes, uncomment each in turn to see what they look like.

%\usetheme{default}
%\usetheme{AnnArbor}
%\usetheme{Antibes}
%\usetheme{Bergen}
%\usetheme{Berkeley}
%\usetheme{Berlin}
%\usetheme{Boadilla}
%\usetheme{CambridgeUS}
%\usetheme{Copenhagen}
%\usetheme{Darmstadt}
\usetheme{Dresden}
%\usetheme{Frankfurt}
%\usetheme{Goettingen}
%\usetheme{Hannover}
%\usetheme{Ilmenau}
%\usetheme{JuanLesPins}
%\usetheme{Luebeck}
%\usetheme{Madrid}
%\usetheme{Malmoe}
%\usetheme{Marburg}
%\usetheme{Montpellier}
%\usetheme{PaloAlto}
%\usetheme{Pittsburgh}
%\usetheme{Rochester}
%\usetheme{Singapore}
%\usetheme{Szeged}
%\usetheme{Warsaw}

% As well as themes, the Beamer class has a number of color themes
% for any slide theme. Uncomment each of these in turn to see how it
% changes the colors of your current slide theme.

%\usecolortheme{albatross}
%\usecolortheme{beaver}
%\usecolortheme{beetle}
%\usecolortheme{crane}
%\usecolortheme{dolphin}
%\usecolortheme{dove}
%\usecolortheme{fly}
%\usecolortheme{lily}
%\usecolortheme{orchid}
%\usecolortheme{rose}
%\usecolortheme{seagull}
%\usecolortheme{seahorse}
%\usecolortheme{whale}
%\usecolortheme{wolverine}

%\setbeamertemplate{footline} % To remove the footer line in all slides uncomment this line
%\setbeamertemplate{footline}[page number] % To replace the footer line in all slides with a simple slide count uncomment this line

%\setbeamertemplate{navigation symbols}{} % To remove the navigation symbols from the bottom of all slides uncomment this line
}

\usepackage{graphicx} % Allows including images
\usepackage{booktabs} % Allows the use of \toprule, \midrule and \bottomrule in 
%tables
\usepackage{listings}
\usepackage{hyperref}
\usepackage{subfig}
\usepackage[export]{adjustbox}
\usepackage{wrapfig}

%----------------------------------------------------------------------------------------
%	TITLE PAGE
%----------------------------------------------------------------------------------------

\title[Lecture 4]{Advanced R Programming - Lecture 4} % The short title 
%appears at the bottom of every slide, the full title is only on the title page

\author{Leif Jonsson} % Your name
\institute[STIMA LiU] % Your institution as it will appear on the bottom of 
%every 
%slide, may be shorthand to save space
{
Link\"{o}ping University \\ % Your institution for the title page
\medskip
\textit{leif.jonsson@ericsson.com\\leif.r.jonsson@liu.se} % Your email address
}
\date{\today} % Date, can be changed to a custom date

\addtobeamertemplate{navigation symbols}{}{ \hspace{1em}    
	\usebeamerfont{footline}%
	\insertframenumber / \inserttotalframenumber }

\begin{document}

\begin{frame}
\titlepage % Print the title page as the first slide
\end{frame}

\begin{frame}
\frametitle{Today} % Table of contents slide, comment this block out to remove 
%it
\tableofcontents % Throughout your presentation, if you choose to use \section{} and \subsection{} commands, these will automatically be printed on this slide as an overview of your presentation
\end{frame}

%----------------------------------------------------------------------------------------
%	PRESENTATION SLIDES
%----------------------------------------------------------------------------------------

\begin{frame}
	\Huge{\centerline{Questions since last time?}}
\end{frame}

\begin{frame}
	\frametitle{Big Bang Theory!}
	\begin{center}
		\begin{figure}[!ht]
			\includegraphics[scale=0.40]{figures/sheldon-game}
			\caption{Rock-paper-scissors according to Sheldon!}
			\label{fig:pkgns}
		\end{figure}
	\end{center}
\end{frame}

\defverbatim[colored]\lstSheldon{
	\begin{lstlisting}[language=R,basicstyle=\ttfamily\footnotesize,keywordstyle=\color{red}]
sheldon_game <- function(player1, player2){
  alt <- c("rock", "lizard", "spock", "scissors", "paper")
  stopifnot(player1 %in% alt, player2 %in% alt)
  alt1 <- which(alt %in% player1)
  alt2 <- which(alt %in% player2)

  if(any((alt1 + c(1,3)) %% 5 == alt2)) {
	return("Player 1 wins!")
  } else {
	return("Player 2 wins!")
  }
  return("Draw!")
}
\end{lstlisting}
}

\begin{frame}
	\frametitle{sheldon\_game}
	\lstSheldon
\end{frame}


%------------------------------------------------
\section{Linear algebra using R}
%------------------------------------------------

\begin{frame}
	\frametitle{Linear algebra in R}
	\begin{center}
	Basics in base \\~\\
	
	Uses LINPACK or LAPACK \\~\\
	
	Extra functionality : Matrix package \\
	(extra LAPACK functionality) \\
	\end{center}
\end{frame}

\defverbatim[colored]\lstLinAlgI{
	\begin{lstlisting}[language=R,basicstyle=\ttfamily\small,keywordstyle=\color{black}]
	# Create matrix
	A <- matrix(1:9,ncol=3)
	
	# Block matrices
	cbind(A,A)
	rbind(A,A)
	
	# Transpose
	t(A)
	
	# Addition and subtraction
	A + A
	A - A
	
	# Matrix multiplication
	A%*%A
	
	# Matrix inversion
	solve(A)
	\end{lstlisting}
}

\begin{frame}
	\frametitle{Linear algebra}
	\lstLinAlgI
\end{frame}


\defverbatim[colored]\lstLinAlgII{
	\begin{lstlisting}[language=R,basicstyle=\ttfamily\small,keywordstyle=\color{black}]
	# Eigenvalues
	eigen(A)
	
	# Determinants
	det(A)
	
	# Matrix factorization
	svd(A)
	qr(A)
	
	# Cholesky decomposition
	chol(A)
	\end{lstlisting}
}

\begin{frame}
	\frametitle{Linear algebra}
	\lstLinAlgII
\end{frame}

%------------------------------------------------
\section{Dynamic reporting with knitr and R-markdown}
%------------------------------------------------

\begin{frame}
	\frametitle{Donald E. Knuth, Literate Programming, 1984}
Let us change our traditional attitude to the construction of programs: Instead 
of imagining that our main task is to instruct a computer what to do, let us 
concentrate rather on explaining to humans what we want the computer to do.\\
- Donald E. Knuth, Literate Programming, 1984
\end{frame}

\begin{frame}
	\frametitle{Background}
	\begin{center}
		Reproducible research \\~\\
		
		Literate programming \\~\\
		
		Dynamic (repeated) reports \\~\\
		
		(Tutorials)
	\end{center}
\end{frame}

\begin{frame}
	\frametitle{markdown}
	\begin{center}
		\begin{figure}[!ht]
			\includegraphics[scale=0.05]{figures/Markdown-mark}
			\label{fig:M}
		\end{figure}
	\end{center}
	\begin{center}
		simple markup language \\~\\
		
		alternative to HTML (and LaTeX) \\~\\
		
		developed further by R-studio  \\
		(see coursepage)
	\end{center}
\end{frame}


\begin{frame}
	\frametitle{knitr + md = rmd}
	Add R to markdown
\end{frame}

\begin{frame}
	\frametitle{knitr + md = rmd}
	Add R to markdown
	\begin{figure}
		\subfloat[.rmd]{\includegraphics[scale=0.05,valign=m]{figures/rmd}}\qquad
		\caption{Flow}
		\label{fig:1}
	\end{figure}
\end{frame}

\begin{frame}
	\frametitle{knitr + md = rmd}
	Add R to markdown
	\begin{figure}
		\subfloat[.rmd]{\includegraphics[scale=0.05,valign=m]{figures/rmd}}\qquad
		\subfloat[knitr]{\includegraphics[scale=0.6,valign=m]{figures/right-arrow}}\qquad
		\caption{Flow}
		\label{fig:1}
	\end{figure}
\end{frame}

\begin{frame}
	\frametitle{knitr + md = rmd}
	Add R to markdown
	\begin{figure}
		\subfloat[.rmd]{\includegraphics[scale=0.05,valign=m]{figures/rmd}}\qquad
		\subfloat[knitr]{\includegraphics[scale=0.6,valign=m]{figures/right-arrow}}\qquad
		\subfloat[markdown]{\includegraphics[scale=0.03,valign=m]{figures/Markdown-mark}}\qquad
		\caption{Flow}
		\label{fig:1}
	\end{figure}
\end{frame}

\begin{frame}
	\frametitle{knitr + md = rmd}
	Add R to markdown
	\begin{figure}
		\subfloat[.rmd]{\includegraphics[scale=0.05,valign=m]{figures/rmd}}\qquad
		\subfloat[knitr]{\includegraphics[scale=0.6,valign=m]{figures/right-arrow}}\qquad
		\subfloat[markdown]{\includegraphics[scale=0.03,valign=m]{figures/Markdown-mark}}\qquad
		\subfloat[pandoc]{\includegraphics[scale=0.4,valign=m]{figures/multi-formats}}\qquad
		\caption{Flow}
		\label{fig:1}
	\end{figure}
\end{frame}

%------------------------------------------------
\section{ggplot2}
%------------------------------------------------

\begin{frame}
	\frametitle{ggplot2}
	\begin{center}
		popular visualization package \\~\\
		
		"The grammar of graphics" \\
		- the language of visualization \\~\\
		
		flexible \\~\\
		
		\href{http://shiny.stat.ubc.ca/r-graph-catalog/}{ggplot examples}
	\end{center}
\end{frame}

\begin{frame}
	\frametitle{the grammar}
	\begin{center}
		Create a graph layer by layer \\~\\
		
		Store as object (print to plot) \\~\\
		
		Three (main) parts: \\~\\
		
	\begin{table}
		\begin{tabular}{l l}
			\texttt{data} & The data to visualize (data.frame) \\
			\texttt{geom} & The geometric representation of data \\
			\texttt{aes} & The mapping of colors/shape to data \\
		\end{tabular}
	\end{table}
	\end{center}
\end{frame}

\begin{frame}
	\frametitle{geom}
	\begin{center}
		\begin{table}
			\begin{tabular}{l l}
				\texttt{geom\_point} & Scatterplots \\
				\texttt{geom\_line} & Lineplots \\
				\texttt{geom\_boxplot} & Boxplot \\
				\texttt{geom\_histogram} & Histograms \\
				\texttt{geom\_bar} & Barchart \\
			\end{tabular}
		\end{table}
	\end{center}
\end{frame}

\begin{frame}
	\frametitle{aes}
	\begin{center}
		\begin{table}
			\begin{tabular}{c}
				\texttt{x} \\
				\texttt{y} \\
				\texttt{size} \\
				\texttt{color} \\
				\texttt{shape} \\
			\end{tabular}
		\end{table}
	\end{center}
\end{frame}

\begin{frame}
	\frametitle{Special \texttt{aes}}
	\begin{center}
		\begin{table}
			\begin{tabular}{l l}
				\texttt{geom} & Special \texttt{aes} \\
				\hline
				\texttt{geom\_point} & point shape, point size \\
				\texttt{geom\_line} & line type, line size \\
				\texttt{geom\_bar} & y min, y max, fill color, outline color \\
			\end{tabular}
		\end{table}
	\end{center}
\end{frame}

\defverbatim[colored]\lstGGPlotI{
	\begin{lstlisting}[language=R,basicstyle=\ttfamily\small,keywordstyle=\color{black}]
	library(ggplot2)
	
	# Preprocessing
	data(Nile)
	Nile <- as.data.frame(Nile) 
	colnames(Nile) <- "level"
	Nile$years <- 1871:1970
	Nile$period <- "- 1900" 
	Nile$period[Nile$years >= 1900] <- "1900 - 1945"
	Nile$period[Nile$years > 1945] <- "1945 +" 
	Nile$period <- as.factor(Nile$period)
	\end{lstlisting}
}

\begin{frame}
	\frametitle{GGPlot2: Example}
	\lstGGPlotI
\end{frame}

\defverbatim[colored]\lstGGPlotII{
	\begin{lstlisting}[language=R,basicstyle=\ttfamily\small,keywordstyle=\color{black}]
	pl <- 
		ggplot(data=Nile) + 
		aes(x=years, y=level) + 
		geom_point()
	pl
	\end{lstlisting}
}

\begin{frame}
	\frametitle{GGPlot2: geom\_point}
	\lstGGPlotII
	\begin{center}
		\begin{figure}[!ht]
			\includegraphics[scale=0.40]{figures/Rplot1}
			\label{fig:gg1}
		\end{figure}
	\end{center}
\end{frame}

\defverbatim[colored]\lstGGPlotIII{
	\begin{lstlisting}[language=R,basicstyle=\ttfamily\small,keywordstyle=\color{black}]
	pl <- 
		ggplot(data=Nile) + 
		aes(x=years, y=level) + 
		geom_line()
	pl
	\end{lstlisting}
}

\begin{frame}
	\frametitle{GGPlot2: geom\_line}
	\lstGGPlotIII
	\begin{center}
		\begin{figure}[!ht]
			\includegraphics[scale=0.40]{figures/Rplot2}
			\label{fig:gg2}
		\end{figure}
	\end{center}
\end{frame}

\defverbatim[colored]\lstGGPlotIV{
	\begin{lstlisting}[language=R,basicstyle=\ttfamily\small,keywordstyle=\color{black}]
	pl <- 
		ggplot(data=Nile) + 
		aes(x=years, y=level, color=period) + 
		geom_line(aes(type=period)) + 
		geom_point(aes(shape=period))
	pl
	\end{lstlisting}
}

\begin{frame}
	\frametitle{GGPlot2: geom\_point + geom\_line + colors!}
	\lstGGPlotIV
	\begin{center}
		\begin{figure}[!ht]
			\includegraphics[scale=0.40]{figures/Rplot3}
			\label{fig:gg3}
		\end{figure}
	\end{center}
\end{frame}

\defverbatim[colored]\lstGGPlotIV{
	\begin{lstlisting}[language=R,basicstyle=\ttfamily\small,keywordstyle=\color{black}]
	pl + theme_bw()
	\end{lstlisting}
}

\begin{frame}
	\frametitle{GGPlot2: use BW theme}
	\lstGGPlotIV
	\begin{center}
		\begin{figure}[!ht]
			\includegraphics[scale=0.40]{figures/Rplot4}
			\label{fig:gg3}
		\end{figure}
	\end{center}
\end{frame}

%------------------------------------------------
\section{Object orientation}
%------------------------------------------------

\begin{frame}
	\frametitle{Object orientation}
	\begin{center}
		Programming paradigm \\~\\
		
		Mutable states \\~\\
		
		Key abstraction is "an object" \\~\\
		
		R is \textit{not} purely object oriented \\~\\
	\end{center}
\end{frame}

\begin{frame}
	\frametitle{Object orientation}

	\begin{figure}
		\subfloat{\includegraphics[scale=0.4,valign=m]{figures/class}}\qquad
		\subfloat{\includegraphics[scale=0.4,valign=m]{figures/constructor}}\qquad
		\subfloat{\includegraphics[scale=0.4,valign=m]{figures/instance}}\qquad
		\subfloat{\includegraphics[scale=0.4,valign=m]{figures/methods}}\qquad
		\label{fig:1}
	\end{figure}
\end{frame}

\begin{frame}
	\frametitle{Object orientation}
	\textbf{Fields} \\
	currency (12/24) : class variable \\
	current\_amount : object variable \\
	no\_withdraws : object variable \\~\\
	
	\textbf{Methods} \\
	insert() \\
	withdraw()
\end{frame}

\begin{frame}
	\frametitle{Inheritance}
	
	\begin{figure}
		\subfloat{\includegraphics[scale=0.4,valign=m]{figures/savings-account}}\qquad
		\subfloat{\includegraphics[scale=0.4,valign=m]{figures/inherit}}\qquad
		\subfloat{\includegraphics[scale=0.42,valign=m]{figures/account}}\qquad
		\label{fig:1}
	\end{figure}
\end{frame}

\begin{frame}
	\frametitle{Object orientation in R}
	\begin{center}
		\begin{table}
			\begin{tabular}{p{3cm} p{3cm} p{3cm}}
				S3 &  &  \\
				\hline
				\hline
				Simple &  &  \\
				\hline
				Methods belongs to functions &  & \\
				\hline
			\end{tabular}
		\end{table}
	\end{center}
\end{frame}

\begin{frame}
	\frametitle{Object orientation in R}
	\begin{center}
		\begin{table}
			\begin{tabular}{p{3cm} p{3cm} p{3cm}}
				S3 & S4 &  \\
				\hline
				\hline
				Simple & More formal &  \\
				\hline
				Methods belongs to functions & Methods belongs to functions & 
				 \\
				\hline
				& @Fields & \\
				\hline
				& Parents & \\
			\end{tabular}
		\end{table}
	\end{center}
\end{frame}

\begin{frame}
	\frametitle{Object orientation in R}
	\begin{center}
		\begin{table}
			\begin{tabular}{p{3cm} p{3cm} p{3cm}}
				S3 & S4 & RC \\
				\hline
				\hline
				Simple & More formal & Latest (R 2.12) \\
				\hline
				Methods belongs to functions & Methods belongs to functions & 
				no copy-on-modify \\
				\hline
				& @Fields & Methods belongs to objects \\
				\hline
				& Parents & Objects have Fields and methods \$ \\
			\end{tabular}
		\end{table}
	\end{center}
\end{frame}

\defverbatim[colored]\lstSIIII{
	\begin{lstlisting}[language=R,basicstyle=\ttfamily\small,keywordstyle=\color{black}]
	# Create object
	x <- 1:100
	class(x) <- "my_numeric"
	\end{lstlisting}
}

\defverbatim[colored]\lstSIIIII{
	\begin{lstlisting}[language=R,basicstyle=\ttfamily\small,keywordstyle=\color{black}]
	# Create generic function
	f <- function(x) UseMethod("f")
	\end{lstlisting}
}

\defverbatim[colored]\lstSIIIIII{
	\begin{lstlisting}[language=R,basicstyle=\ttfamily\small,keywordstyle=\color{black}]
	# Create method
	print.my_numeric <- function(x, ...){
		cat("This is my numeric vector.")
	}
	\end{lstlisting}
}

\begin{frame}
	\frametitle{S3}
	\lstSIIII
\end{frame}

\begin{frame}
	\frametitle{S3}
	\lstSIIII
	\lstSIIIII
\end{frame}

\begin{frame}
	\frametitle{S3}
	\lstSIIII
	\lstSIIIII
	\lstSIIIIII
\end{frame}

\defverbatim[colored]\lstRC{
	\begin{lstlisting}[language=R,basicstyle=\ttfamily\small,keywordstyle=\color{black}]
# Create object with fields and methods
Account <- setRefClass("Account",
	fields = list(balance = "numeric"),
	methods = list(
		withdraw = function(x) {
			balance <<- balance - x
		},
		deposit = function(x) {
			balance <<- balance + x
		}
	)
)
	\end{lstlisting}
}

\begin{frame}
	\frametitle{RC}
	\lstRC
	\texttt{object\$copy()}
\end{frame}



%------------------- THE END ---------------------------------------------------

\begin{frame}
\Huge{\centerline{The End... for today.}}
\Huge{\centerline{Questions?}}
\Huge{\centerline{See you next time!}}
\end{frame}

%-------------------------------------------------------------------------------

\end{document} 